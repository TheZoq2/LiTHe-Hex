%% TODO%% TODO%% TODO%% TODO%% TODO%% TODO%% TODO%% TODO%% TODO%% TODO%% TODO           
% Milstolpar, material, utveklingsmetodik, tidsplan, simuleringar
% 
\documentclass[a4paper,titlepage,12pt]{article}
\usepackage[utf8]{inputenc} %Make sure all UTF8 characters work in the document
\usepackage{color}
\usepackage{graphicx}
\usepackage{titling}
\usepackage[titletoc,title]{appendix}
\usepackage{tabularx}
\usepackage{longtable}
\usepackage[yyyymmdd]{datetime}
\usepackage[figurename=Figur]{caption}
\usepackage{pbox}
\usepackage{booktabs}
\usepackage[parfill]{parskip}

%Set page size
\usepackage{geometry}
\geometry{margin=3cm}

\renewcommand{\dateseparator}{-}
\renewcommand{\contentsname}{Innehållsförteckning}
\renewcommand{\appendixname}{Bilaga}

%%%%%%%%%%%%%%%%%%%%%%%%%%%%%%%
% Header and footer
%%%%%%%%%%%%%%%%%%%%%%%%%%%%%%%
\usepackage{fancyhdr}
\pagestyle{fancy}

\lhead{\includegraphics[width=0.15\linewidth]{../images/logo_full.png}}
\chead{Projektplan för sexbent robot}
\rhead{\today}
\setlength\headheight{26pt} 

\lfoot{TSEA29 --- KMM \\ LIPS Projektplan}
\rfoot{Grupp 9 \\ LiTHe Hex}

\pretitle{%
    \begin{center}
        \LARGE
        \includegraphics[width=6cm]{../images/logo_full.png}\\[\bigskipamount]
}

\posttitle{\end{center}}

\newcounter{milNr}
\setcounter{milNr}{0}
\newcommand{\nextMilNr}{\stepcounter{milNr}\arabic{milNr}}

\newcounter{bpNr}
\setcounter{bpNr}{-1}
\newcommand{\nextBPNr}{\stepcounter{bpNr}\arabic{bpNr}}

\newcounter{aktNr}
\setcounter{aktNr}{0}
\newcommand{\nextAktNr}{\stepcounter{aktNr}\arabic{aktNr}}

\begin{document}
    \title{\LARGE
        \textbf{Projektplan för sexbent robot} \\
        \vspace*{0.5\baselineskip}
        \large
        Redaktör Malcolm Vigren \\
        Grupp 9 \\
        \small
        \vspace*{0.5\baselineskip}
        Version 0.1}

    \date{\today}

		\maketitle
	
	\newpage
	
	\begin{center}

		%%%%%%%%%%%%%%%%%%%%%%%%%%%%%%%%%%%%%%%%%%%%%%%%%%%%%%%%%%%%%%%%%%%%%%%%%%%%%%%%%
		%						Medlemmar
		%%%%%%%%%%%%%%%%%%%%%%%%%%%%%%%%%%%%%%%%%%%%%%%%%%%%%%%%%%%%%%%%%%%%%%%%%%%%%%%%%

		\section*{Projektidentitet}
		Grupp 9, Ht 2016, LiTHe Hex

		Linköpings Tekniska Högskola, ISY

		\renewcommand*{\arraystretch}{1.4}
		\begin{longtable}[c]{ l l l }
			\textbf{Namn} & \textbf{Ansvar} & \textbf{E-post} \\ \midrule
			Emil Segerbäck & & emise935@student.liu.se \\ \midrule
			Frans Skarman & Dokumentansvarig & frask812@student.liu.se \\ \midrule
			Hannes Tuhkala & & hantu447@student.liu.se \\ \midrule
			Malcolm Vigren & Projektledare & malvi108@student.liu.se \\ \midrule
			Noak Ringman &  & noari093@student.liu.se \\ \midrule
			Olav Övrebö &  & olaov121@student.liu.se \\ \midrule
			Robin Sliwa &  & robsl733@student.liu.se \\
		\end{longtable}

		\centering
		\textbf{Kursansvarig}: Tomas Svensson Rum 3B:528 013--28 13 68 tomas.svensson@liu.se

		\newpage
		\tableofcontents
		\newpage


		%%%%%%%%%%%%%%%%%%%%%%%%%%%%%%%%%%%%%%%%%%%%%%%%%%%%%%%%%%%%%%%%%%%%%%%%%%%%%%%%%
		%						Historik
		%%%%%%%%%%%%%%%%%%%%%%%%%%%%%%%%%%%%%%%%%%%%%%%%%%%%%%%%%%%%%%%%%%%%%%%%%%%%%%%%%

		\section*{Dokumenthistorik}
		\renewcommand*{\arraystretch}{1.4}
		\begin{longtable}[c]{ l l l l l }
			\textbf{Version} & \textbf{Datum} & \textbf{Utförda förändringar} 
			& \textbf{Utförda av} & \textbf{Granskad} \\ \midrule

			0.1 & 2016--09--20 & Första utkastet & Projektgruppen & \\
		\end{longtable}
	\end{center}

	\newpage

	\section{Beställare}
	Beställare är Tomas Svensson, lektor vid Linköpings tekiska högskola. \\
  Kontaktdata: Rum 3B:528 013–28 13 68 tomas.svensson@liu.se

	%%%%%%%%%%%%%%%%%%%%%%%%%%%%%%%%%%%%%%%%%%%%%%%%%%%%%%%%%%%%%%%%%%%%%%%%%%%%%%%%%
	%						Översikt
	%%%%%%%%%%%%%%%%%%%%%%%%%%%%%%%%%%%%%%%%%%%%%%%%%%%%%%%%%%%%%%%%%%%%%%%%%%%%%%%%%

	\newpage
	\section{Översiktlig beskrivning av projektet}
	Text

	\subsection{Syfte och mål}
	Syftet och målet med projektet är att utveckla en sexbent robot som själv
    kan navigera sig ut ur en labyrint. I labyrinten bör roboten även kunna ta
    sig över hinder för att komma vidare.

    % TODO varför ska vi lägga till "specifika syften som berör projektet och
    % dess medlemmar"???!!??!
	
	
	\subsection{Leveranser}
	De dokument som ska levereras till kunden är: projektplan, tidplan,
    systemskiss, designspecifikation, teknisk dokumentation och
    användarhandledning. Slutleveransen består av en presentation av projektet,
    demonstration av roboten i autonomnt och manuellt läge i form av en tävling,
    samt överlämning av kod, hårdvara och dokumentation.
	
	
	\subsection{Begränsningar}
	Roboten behöver inte klara mer avancerade former på labyrinten än de som
    beskrivs i Ban- och regelspecifikationen.
	
	
	\section{Fasplan}
	Här var det text här var det text här var det text
	här var det text här var det text här var det text
	här var det text här var det text här var det text.
	
	% \subsection{Före projektstart}
    % Före projektstarten har en grupp bildats med 7 medlemmar, där varje medlem
    % har tilldelats ett eller flera ansvarsområden inom projektet. En
    % kravspecifikation har skrivits enligt projektdirektivet och godkänts av
    % beställaren. En projektplan, tidsplan och systemskiss har skrivits och
    % godkänts av beställaren.
	
	
	\subsection{Under projektet}
    %TODO add simuleringar???
	Under projektet ska en designspecifikation skrivas med detaljerad
    information om hur roboten ska implementeras. Efter att
    designspecifikationen är klar påbörjas implementationen av systemet.
    Enhetstester ska konstrueras i takt med att testbara delar av systemet har
    implementerats. Teknisk dokumentation för varje delsystemet ska skrivas
    i takt med att implementationsbeslut utförs.

    Hela gruppen ska ha möten minst en gång per vecka för att presentera
    varandras arbete sedan förra mötet, och för att kontrollera att projektet
    följer planen. Dessa möten blir också tillfällen att diskutera problem som
    uppstår med hela gruppen.

    Om ett visst krav i kravspecifikationen inte kan implementeras ska ett möte
    med beställaren anordnas för omförhandling av kravet i fråga.
	
	\subsection{Efter projektet}
    Projektet avslutas med slutleverans efter att alla krav i
    kravspecifikationen är uppfyllda. Efter leveransen görs en efterstudie.
	
	
	\section{Organisationsplan för hela projektet}
    %I detta avsnitt beskrivs hur projektet ska organiseras.
	Projektrollerna som ingår i detta projekt är: beställare/kund, handlerare,
	projektledare och projektmedlemmar. Alla projektmedlemmar har kontakt med
    handledaren, men det är projektledaren som sköter den huvudsakliga
    kontakten med beställaren, se figur \ref{fig:organization}.

    \begin{figure}[h!]
        \begin{center}
		\includegraphics[width=0.8\linewidth]{images/projectroles.png}
        \caption{Projektets organisation \label{fig:organization}}
        \end{center}
	\end{figure}
	 
	\subsection{Villkor för samarbetet inom projektgruppen}
    % TODO create appendix
    Se Bilaga \ref{app:gruppkontrakt} - Gruppkontrakt.

	
	\subsection{Definition av arbetsinnehåll och ansvar}
    Beställarens huvudsakliga uppgifter är att fatta beslut
    om fortsättning av projektet vid beslutspunkter, godkänna
	ändringar under projektets gång, läsa statusrapporter samt godkännande av
	projektets avslut. Arbetsuppgifterna för projektledaren är, utöver
    arbetsuppgifterna för resten av projektmedlemmarna, att fördela
	arbetsuppgifter, planera och leda projektarbetet, se till att projektets mål
	nås samt motivera projektmedlemmarna.

    Kommunikationen mellan projektmedlemmar kommer att ske genom möten, en
    gemensam kalender samt en gruppchatt.
	
    För information om hur arbetet på delsystemen ska fördelas, se Bilaga
    \ref{app:tidsplan} - Tidsplan.
	
	\section{Dokumentplan}
    Dokumenten lagras på GitHub och Git används för versionshantering av
    dokumenten, som skrivs i LaTeX. Alla projektmedlemmar har tillgång till
    samtliga dokument. All dokumentation, förutom kommentarer och doc-strängar
    i kod, är skriven på svenska.

    Stora ändringar ger inkrementering av heltalssiffran i versionsnumrena,
    medan små ändringar ger inkrementation av decimalsiffran.
	
    % FIXME kontrollera ansvarig och sånt
    % TODO lägg till eventuella fler interna dokument
    \begin{longtable}[l]{ p{2.5cm} p{2.1cm} p{4cm} p{2.2cm} l }
        \textbf{Dokument} & \textbf{Ansvarig/ godkänns av} & \textbf{Syfte} & \textbf{Distribueras till} & \textbf{Färdigdatum} \\ \midrule
        
        Kravspeci-fikation & Beställare & Definierar alla krav på
        systemet & Beställare & 2016--09--13 \\ \midrule

        Projektplan & Beställare & <SOMETHING> & Beställare &
        2016--09--29 \\ \midrule

        Tidplan & Beställare & Planering för hur arbetstid ska
        fördelas. & Beställare & 2016--09--29  \\ \midrule
        
        Systemskiss & Beställare & Grov skiss och idéer för 
        implementation av systemet & Beställare & 2016--09--29 \\ \midrule

        Designspeci-fikation & Handledare & Detaljerad beskrivning av
        implementationen av systemet. & Handledare & 2016--11--04 \\ \midrule

        Tidrapporter & Beställare & Rapportering av använd arbetstid &
        Beställare & Veckovis        \\ \midrule

        Statusrapporter & Beställare & Rapportering av nuvarande framteg &
        Beställare & Veckovis \\ \midrule

        Teknisk dokumentation & Beställare & Teknisk beskriving av
        implementationen av den färdiga produkten. & Beställare &
        2016--12--17 \\ \midrule

        Användar- handledning & Beställare & En manual för användning av
        produkten & Beställare & 2016--12--17 \\ \midrule

        Efterstudie & Beställare & Utvärdering av projektet &
        Beställare & 2016--12--22  \\ \midrule
    \end{longtable}
	
	
	\section{Utvecklingsmetodik}
	Här var det text här var det text här var det text
	här var det text här var det text här var det text
	här var det text här var det text här var det text.
	
	
	\section{Utbildningsplan}

    De gruppmedlemmar som arbetar på ett visst delsystem bör vara fullt
    utbildade i alla tekniska detaljer som ingår i implementationen av det
    delsystemet. Gruppmedlemmar som arbetar på andra system behöver inte lika
    stor insikt i de tekniska detaljerna, men bör
    informera sig om gränssnitten mellan sin enhet och andra delsystem, samt väldigt 
    grundläggande information om hur enheterna är uppbyggda och deras syften.

    Varje delgrupp är ansvarig i att utbilda sig inom de kunskaper som krävs
    för utveckling av sina respekive delsystem.

	\section{Rapporteringsplan}
	Tids- och statusrapporter ska skrivas varje vecka. Detta görs lämpligen
    efter veckomötet av den utvalda sekreteraren för det mötet.
	
	\section{Mötesplan}
    Minst en gång per vecka ska ett möte hållas, där alla gruppmedlemmar
    redovisar för vad de har gjort, vilka problem de har löst, samt hur de
    tänker fortsätta sitt arbete. Mötena ska även användas för diskussion av
    problem gällande de olika delsystemen eller hela systemet. Dessa möten tar
    ungefär 1 timme, och vid varje möte utses en sekreterare som får skriva en
    statusrapport från det som togs upp på mötet. Mötena ska ske i början av
    varje vecka.
	
	
	\section{Resursplan}
	Här var det text här var det text här var det text
	här var det text här var det text här var det text
	här var det text här var det text här var det text.
	
	
	\subsection{Personer}
    Alla 7 projektmedlemmar förväntas arbeta ungefär lika mycket under alla
    projektfaser. Projektgruppen har tillgång till en handlerare.
	
	\subsection{Material}
	Vi behöver följande material:
    \begin{itemize}
            \item Chassi för hexapod-robot inklusive servon
            \item 2 AVR-processorer
            \item 1 Raspberry Pi-enkortsdator (3:e generationen)
            \item Några avståndssensorer %TODO SENSORS????
            % TODO not done
    \end{itemize}
	
	
	\subsection{Lokaler}
    Muxen kommer användas för kodning och testning med robot. Grupprum kommer
    användas för kodning, design och möten.
	
	
	\subsection{Ekonomi}
    Efter godkänd projektplan får maximalt 1120 arbetstimmar användas på projektet.
	
	
	\section{Milstolpar och beslutspunkter}
	Här var det text här var det text här var det text
	här var det text här var det text här var det text
	här var det text här var det text här var det text.
	
	
	\subsection{Milstolpar}
	\begin{longtable}[c]{ c l c}
		\textbf{Nr} & \textbf{Beskrivning} & \textbf{Datum} \\ \midrule
		\nextMilNr{} & Kravspecifikationen är klar & 2016--09--13 \\ \midrule
		\nextMilNr{} & Projektplan, systemskiss och tidsplan är klar & -- \\ \midrule
		\nextMilNr{} & Designspecifikation är klar & -- \\ \midrule
	\end{longtable}
	
	
	\subsection{Beslutspunkter}
	\renewcommand*{\arraystretch}{1.4}
	\begin{longtable}[c]{ c l c}
		\textbf{Nr} & \textbf{Beskrivning} & \textbf{Datum} \\ \midrule
		\nextBPNr{} & Godkännande av projektdirektiv, beslut att starta förstudie & 2016--09--02 \\ \midrule
		\nextBPNr{} & Godkännande av kravspecifikation, beslut att starta förberedelsefasen & 2016--09--13 \\ \midrule
		\nextBPNr{} & Godkännande av projektplanering, beslut att starta utförandefasen & 2016--09--29 \\ \midrule
		\nextBPNr{} & Godkännande av designspecifikation, beslut att fortsätta
        utförandefasen & -- \\ \midrule
		\nextBPNr{} & Används ej & -- \\ \midrule
		\nextBPNr{} & Godkännande av produktens funktionalitet, beslut att leverera & -- \\ \midrule
		\nextBPNr{} & Godkännande av leverans, beslut att upplösa projektgruppen & -- \\ \midrule
	\end{longtable}
	
	
	\section{Aktiviteter}
	\renewcommand*{\arraystretch}{1.4}
	\begin{longtable}[c]{ c p{3cm} p{6cm} p{3cm} p{2cm}}
		\textbf{Nr} & \textbf{Aktivitet} & \textbf{Beskrivning} & \textbf{Beroende av aktivitet nr} & \textbf{Beräknad tid tim} \\ \midrule
		\nextAktNr{} & Inverse Kinematics & -- & -- & -- \\ \midrule
		\nextAktNr{} & Servokontroll & -- & -- & -- \\ \midrule
		\nextAktNr{} & Gångstil & -- & -- & -- \\ \midrule
		\nextAktNr{} & Kommunikation & -- & -- & -- \\ \midrule
		\nextAktNr{} & Golvhöjdsdetektion & -- & -- & -- \\ \midrule
		\nextAktNr{} & Stötdetektion & -- & -- & -- \\ \midrule
		\nextAktNr{} & Hindergång & -- & -- & -- \\ \midrule
		\nextAktNr{} & Kommunikation med motorikenhet & -- & -- & -- \\ \midrule
		\nextAktNr{} & Detektera återvändsgränser & -- & -- & -- \\ \midrule
		\nextAktNr{} & Kommunikation med sensorenhet & -- & -- & -- \\ \midrule
		\nextAktNr{} & Kommunikation med GUI & -- & -- & -- \\ \midrule
		\nextAktNr{} & Följa korridor & -- & -- & -- \\ \midrule
		\nextAktNr{} & Hinderdetektion & -- & -- & -- \\ \midrule
		\nextAktNr{} & Hinderhantering & -- & -- & -- \\ \midrule
		\nextAktNr{} & Beslutsfattning & -- & -- & -- \\ \midrule
		\nextAktNr{} & GUI kommandotolkning & -- & -- & -- \\ \midrule
		\nextAktNr{} & Telemetri & -- & -- & -- \\ \midrule
		\nextAktNr{} & Styrning & -- & -- & -- \\ \midrule
		\nextAktNr{} & Webbserver & -- & -- & -- \\ \midrule
		\nextAktNr{} & Frontend & -- & -- & -- \\ \midrule
		\nextAktNr{} & Joystick control & -- & -- & -- \\ \midrule
		\nextAktNr{} & Debug log & -- & -- & -- \\ \midrule
		\nextAktNr{} & Läsning av IR-sensor & -- & -- & -- \\ \midrule
		\nextAktNr{} & Kommunikation med centralenhet & -- & -- & -- \\ \midrule
		\nextAktNr{} & Läsning av gyro-sensor & -- & -- & -- \\ \midrule
		\nextAktNr{} & Läsning av LIDAR & -- & -- & -- \\ \midrule
		\nextAktNr{} & Brusdetektion av IR-sensor & -- & -- & -- \\ \midrule
		\nextAktNr{} & Integral av gyro & -- & -- & -- \\ \midrule
		\nextAktNr{} & Multiplexning A/D omvandlare & -- & -- & -- \\ \midrule
		\nextAktNr{} & Enhetskonvertering & -- & -- & -- \\ \midrule
		\nextAktNr{} & Buffertid & -- & -- & -- \\ \midrule
		\nextAktNr{} & Projektmöten & -- & -- & -- \\ \midrule
		\nextAktNr{} & Milstolpe 1 & -- & -- & -- \\ \midrule
		\nextAktNr{} & Milstolpe 2 & -- & -- & -- \\ \midrule
		\nextAktNr{} & Milstolpe 3 & -- & -- & -- \\ \midrule
		\nextAktNr{} & Milstolpe 4 & -- & -- & -- \\ \midrule
		\nextAktNr{} & Milstolpe 5 & -- & -- & -- \\ \midrule
		\nextAktNr{} & Beslutspunkt 1 & -- & -- & -- \\ \midrule
		\nextAktNr{} & Beslutspunkt 2 & -- & -- & -- \\ \midrule
		\nextAktNr{} & Beslutspunkt 5 & -- & -- & -- \\ \midrule
	\end{longtable}

	
	\section{Tidplan}
	Här var det text här var det text här var det text
	här var det text här var det text här var det text
	här var det text här var det text här var det text.
	
	
%	\section{Förändringsplan}
%	Här var det text här var det text här var det text
%	här var det text här var det text här var det text
%	här var det text här var det text här var det text.
	
	
	\section{Kvalitetsplan}
	Här var det text här var det text här var det text
	här var det text här var det text här var det text
	här var det text här var det text här var det text.
	
	
	\subsection{Granskningar}
	Här var det text här var det text här var det text
	här var det text här var det text här var det text
	här var det text här var det text här var det text.
	
	
	\subsection{Testplan}
	Här var det text här var det text här var det text
	här var det text här var det text här var det text
	här var det text här var det text här var det text.
	
	
%	\section{Riskanalys}
%	Här var det text här var det text här var det text
%	här var det text här var det text här var det text
%	här var det text här var det text här var det text.
	
	
	\section{Prioriteringar}
	Här var det text här var det text här var det text
	här var det text här var det text här var det text
	här var det text här var det text här var det text.
	
	
	\section{Projektavslut}
	Här var det text här var det text här var det text
	här var det text här var det text här var det text
	här var det text här var det text här var det text.

\begin{appendices}
    \section{Gruppkontrakt\label{app:gruppkontrakt}}
    \section{Tidsplan\label{app:tidsplan}}
\end{appendices}
\end{document}
