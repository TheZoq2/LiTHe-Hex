\documentclass[a4paper,titlepage,12pt]{article}
\usepackage[utf8]{inputenc} %Make sure all UTF8 characters work in the document
\usepackage{graphicx}
\usepackage{titling}
\usepackage{tabularx}
\usepackage{longtable}
\usepackage[yyyymmdd]{datetime}
\usepackage[figurename=Figur]{caption}
\usepackage{booktabs}
\usepackage[parfill]{parskip}

%Set page size
\usepackage{geometry}
\geometry{margin=3cm}

\renewcommand{\dateseparator}{-}
\renewcommand{\contentsname}{Innehållsförteckning}
\renewcommand{\tablename}{Tabell}


\usepackage{listings}
\usepackage{color}

\usepackage[colorinlistoftodos,prependcaption,textsize=tiny]{todonotes}

\definecolor{dkgreen}{rgb}{0,0.6,0}
\definecolor{gray}{rgb}{0.5,0.5,0.5}
\definecolor{mauve}{rgb}{0.58,0,0.82}

\lstset{frame=tb,
	language=Python,
	aboveskip=3mm,
	belowskip=3mm,
	showstringspaces=false,
	columns=flexible,
	basicstyle={\small\ttfamily},
	numbers=none,
	numberstyle=\tiny\color{gray},
	keywordstyle=\color{blue},
	commentstyle=\color{dkgreen},
	stringstyle=\color{mauve},
	breaklines=true,
	breakatwhitespace=true,
	tabsize=3
}

%%%%%%%%%%%%%%%%%%%%%%%%%%%%%%%
% Header and footer
%%%%%%%%%%%%%%%%%%%%%%%%%%%%%%%
\usepackage{fancyhdr}
\pagestyle{fancy}

\lhead{\includegraphics[width=0.15\linewidth]{../images/logo_full.png}}
\chead{Efterstudie}
\rhead{\today}
\setlength\headheight{26pt} 

\lfoot{TSEA29 -- KMM \\ LIPS Efterstudie}
\rfoot{Grupp 9 \\ LiTHe Hex}

\newcommand{\itc}{I\textsuperscript{2}C}

\pretitle{%
	\begin{center}
		\LARGE
		\includegraphics[width=6cm]{../images/logo_full.png}\\[\bigskipamount]
	}
	
	\posttitle{\end{center}}

\begin{document}
	\listoftodos
	\title{\LARGE
		\textbf{Efterstudie} \\
		\vspace*{0.5\baselineskip}
		\large
		Redaktör ???? \\
		Grupp 9 \\
		\small
		\vspace*{0.5\baselineskip}
		Version 1.0}
	
	\date{\today}
	
	\maketitle
	
	\newpage
	
	\begin{center}
		
		%%%%%%%%%%%%%%%%%%%%%%%%%%%%%%%%%%%%%%%%%%%%%%%%%%%%%%%%%%%%%%%%%%%%%%%%%%%%%%%%%
		%						Medlemmar
		%%%%%%%%%%%%%%%%%%%%%%%%%%%%%%%%%%%%%%%%%%%%%%%%%%%%%%%%%%%%%%%%%%%%%%%%%%%%%%%%%
		
		\section*{Projektidentitet}
		Grupp 9, Ht 2016, LiTHe Hex
		
		Linköpings Tekniska Högskola, ISY
		
		\renewcommand*{\arraystretch}{1.4}
		\begin{longtable}[c]{ l l l }
			\textbf{Namn} & \textbf{Ansvar} & \textbf{E-post} \\ \midrule
			Emil Segerbäck & & emise935@student.liu.se \\ \midrule
			Frans Skarman & Dokumentansvarig & frask812@student.liu.se \\ \midrule
			Hannes Tuhkala & & hantu447@student.liu.se \\ \midrule
			Malcolm Vigren & Projektledare & malvi108@student.liu.se \\ \midrule
			Noak Ringman &  & noari093@student.liu.se \\ \midrule
			Olav Övrebö &  & olaov121@student.liu.se \\ \midrule
			Robin Sliwa &  & robsl733@student.liu.se \\
		\end{longtable}
		
		\centering
		\textbf{Kursansvarig}: Tomas Svensson Rum 3B:528 013--28 13 68 tomas.svensson@liu.se
		
		\newpage
		\tableofcontents
		\newpage
		
		
		%%%%%%%%%%%%%%%%%%%%%%%%%%%%%%%%%%%%%%%%%%%%%%%%%%%%%%%%%%%%%%%%%%%%%%%%%%%%%%%%%
		%						Historik
		%%%%%%%%%%%%%%%%%%%%%%%%%%%%%%%%%%%%%%%%%%%%%%%%%%%%%%%%%%%%%%%%%%%%%%%%%%%%%%%%%
		
		\section*{Dokumenthistorik}
		\renewcommand*{\arraystretch}{1.4}
		\begin{longtable}[c]{ l l >{\raggedright}p{5cm} >{\raggedright}p{3cm} l }
			\textbf{Version} & \textbf{Datum} & \textbf{Utförda förändringar} 
			& \textbf{Utförda av} & \textbf{Granskad} \\ \midrule
			
			1.0 & 2016--12--20 & Första versionen & Projektgruppen &
			Projektgruppen \\
			
		\end{longtable}
	\end{center}
	
	%%%%%%%%%%%%%%%%%%%%%%%%%%%%%%%%%%%%%%%%%%%%%%%%%%%%%%%%%%%%%%%%%%%%%%%%%%%%%%%%%
	%						Inledning
	%%%%%%%%%%%%%%%%%%%%%%%%%%%%%%%%%%%%%%%%%%%%%%%%%%%%%%%%%%%%%%%%%%%%%%%%%%%%%%%%%
	
	\newpage
	
	\raggedright
	
	\section{Tidsåtgång}
	Nu är det dags för Hannes
	
	\subsection{Arbetsfördelning}
	Nu är det dags för Hannes
	
	\subsection{Tidsåtgång jämfört med planerad tid}
	Nu är det dags för Hannes
	
	\begin{longtable}[c]{l l l }
		\textbf{Fas} & \textbf{Planerad tid i timmar} & \textbf{Använd tid i timmar} \\ \midrule
		Före & wat & wat \\
		Under & 1120 & 1329 \\
		Efter & wat & wat \\
	\end{longtable}
	
	\section{Analys av arbete och problem}
	Nu är det dags för Hannes
	
	\subsection{Vad hände under de olika faserna (bra/dåligt/orsak)?}
	Nu är det dags för Hannes
	
	\subsection{Hur vi arbetade tillsammans (ansvar, beslut, kommunikation etc.)?}
	Nu är det dags för Hannes
	
	\subsection{Hur använde vi projektmodellen?}
	Nu är det dags för Hannes
	
	\subsection{Hur fungerade relationen med beställaren?}
    % TODO review
    Relationen med beställaren har gått bra. Tidrapporter har skickats i tid
    varje vecka och statusrapporter på begäran. 
	
	\subsection{Hur fungerade relationen med handledaren?}
	Nu är det dags för Hannes
	
	\subsection{Tekniska framgångar/problem}
	Beskriv de 3 besvärligaste problemen som ni har haft under projektet. 
	Rangordna problemen där 1 är mest besvärlig i betydelsen tog längst tid att fixa.
	Beskriv för varje problem:

	a. felets typ - ange t.ex.  hårdvarufel, mjukvarufel, systemintegrationsfel 

	b. kort beskrivning av felets symptom 

	c. kort beskrivning av felets orsak t.ex. logiskt fel i programmering, timing-fel, glappkontakt, missförstånd av spec., felaktig design. 

	d. kort beskrivning av hur ni fixade felet. T.ex. hur ni fixade mjukvarubuggen, fick ny hårdvara, virade om, gick runt problemet.

	e. ungefärlig skattning av hur mycket tid som ni ägnade åt felsökningen (dvs. tid som kunde tjänats in om felet inte hade uppstått)
	
    % TODO kanske SPI-bussen, servokommunikation, tornadon, LIDARens I2C?
    % TODO ska vi skriva problem 3 först?

    \subsubsection{Problem 3: Här var det problem}

    Problem     Problem     Problem     Problem     Problem     Problem
    Problem     Problem     Problem     Problem     Problem     Problem 

    \subsubsection{Problem 2: Här var det problem}

    Problem     Problem     Problem     Problem     Problem     Problem
    Problem     Problem     Problem     Problem     Problem     Problem 

    \subsubsection{Problem 1: Instabil SPI-buss}

    Detta hårdvarufel uppkom i slutet av vecka 49. SPI-kommunikationen visade
    sig vara oerhört instabil när både motorikenheten och sensorenheten
    användes tillsammans på bussen. Undersökning med oscilloskåp visade att
    signalerna hade mycket störningar, som gjorde kommunikationen svår,
    särskilt till sensorenheten.

    Lösningen på detta problem var helt enkelt att korta ner kablarna som
    används i SPI-bussen, så att de blev så korta som möjligt.

    Hade korta och mer strukturerat virade kablar använts, hade runt 56 timmar
    sparats. % TODO check

	\section{Måluppfyllelse}
	Nu är det dags för Hannes
	
	\subsection{Vad har uppnåtts?}
	Nu är det dags för Hannes
	
	\subsection{Hur fungerade leveransen?}
	Nu är det dags för Hannes
	
	\subsection{Hur har studiesituationen påverkat projektet?}
	Nu är det dags för Hannes
	
	\section{Sammanfattning}
	Nu är det dags för Hannes
	
	\subsection{De tre viktigaste erfarenheterna}
	Nu är det dags för Hannes
	
	\subsection{Goda råd till de som ska utföra ett liknande projekt}
	Nu är det dags för Hannes
	
\end{document}
