\documentclass[a4paper,titlepage,12pt]{article}
\usepackage[utf8]{inputenc} %Make sure all UTF8 characters work in the document
\usepackage{graphicx}
\usepackage{titling}
\usepackage{tabularx}
\usepackage{longtable}
\usepackage[yyyymmdd]{datetime}
\usepackage[figurename=Figur]{caption}
\usepackage{booktabs}
\usepackage[parfill]{parskip}

%Set page size
\usepackage{geometry}
\geometry{margin=3cm}

\renewcommand{\dateseparator}{-}
\renewcommand{\contentsname}{Innehållsförteckning}
\renewcommand{\tablename}{Tabell}


\usepackage{listings}
\usepackage{color}


\definecolor{dkgreen}{rgb}{0,0.6,0}
\definecolor{gray}{rgb}{0.5,0.5,0.5}
\definecolor{mauve}{rgb}{0.58,0,0.82}

\lstset{frame=tb,
	language=Python,
	aboveskip=3mm,
	belowskip=3mm,
	showstringspaces=false,
	columns=flexible,
	basicstyle={\small\ttfamily},
	numbers=none,
	numberstyle=\tiny\color{gray},
	keywordstyle=\color{blue},
	commentstyle=\color{dkgreen},
	stringstyle=\color{mauve},
	breaklines=true,
	breakatwhitespace=true,
	tabsize=3
}

%%%%%%%%%%%%%%%%%%%%%%%%%%%%%%%
% Header and footer
%%%%%%%%%%%%%%%%%%%%%%%%%%%%%%%
\usepackage{fancyhdr}
\pagestyle{fancy}

\lhead{\includegraphics[width=0.15\linewidth]{../images/logo_full.png}}
\chead{Efterstudie}
\rhead{\today}
\setlength\headheight{26pt} 

\lfoot{TSEA29 -- KMM \\ LIPS Efterstudie}
\rfoot{Grupp 9 \\ LiTHe Hex}

\newcommand{\itc}{I\textsuperscript{2}C}

\pretitle{%
	\begin{center}
		\LARGE
		\includegraphics[width=6cm]{../images/logo_full.png}\\[\bigskipamount]
	}
	
	\posttitle{\end{center}}

\begin{document}
	\title{\LARGE
		\textbf{Efterstudie} \\
		\vspace*{0.5\baselineskip}
		\large
		Redaktör Frans Skarman\\
		Grupp 9 \\
		\small
		\vspace*{0.5\baselineskip}
		Version 1.0}
	
	\date{\today}
	
	\maketitle
	
	\newpage
	
	\begin{center}
		
		%%%%%%%%%%%%%%%%%%%%%%%%%%%%%%%%%%%%%%%%%%%%%%%%%%%%%%%%%%%%%%%%%%%%%%%%%%%%%%%%%
		%						Medlemmar
		%%%%%%%%%%%%%%%%%%%%%%%%%%%%%%%%%%%%%%%%%%%%%%%%%%%%%%%%%%%%%%%%%%%%%%%%%%%%%%%%%
		
		\section*{Projektidentitet}
		Grupp 9, Ht 2016, LiTHe Hex
		
		Linköpings Tekniska Högskola, ISY
		
		\renewcommand*{\arraystretch}{1.4}
		\begin{longtable}[c]{ l l l }
			\textbf{Namn} & \textbf{Ansvar} & \textbf{E-post} \\ \midrule
			Emil Segerbäck & & emise935@student.liu.se \\ \midrule
			Frans Skarman & Dokumentansvarig & frask812@student.liu.se \\ \midrule
			Hannes Tuhkala & & hantu447@student.liu.se \\ \midrule
			Malcolm Vigren & Projektledare & malvi108@student.liu.se \\ \midrule
			Noak Ringman &  & noari093@student.liu.se \\ \midrule
			Olav Övrebö &  & olaov121@student.liu.se \\ \midrule
			Robin Sliwa &  & robsl733@student.liu.se \\
		\end{longtable}
		
		\centering
		\textbf{Kursansvarig}: Tomas Svensson Rum 3B:528 013--28 13 68 tomas.svensson@liu.se
		
		\newpage
		\tableofcontents
		\newpage
		
		
		%%%%%%%%%%%%%%%%%%%%%%%%%%%%%%%%%%%%%%%%%%%%%%%%%%%%%%%%%%%%%%%%%%%%%%%%%%%%%%%%%
		%						Historik
		%%%%%%%%%%%%%%%%%%%%%%%%%%%%%%%%%%%%%%%%%%%%%%%%%%%%%%%%%%%%%%%%%%%%%%%%%%%%%%%%%
		
		\section*{Dokumenthistorik}
		\renewcommand*{\arraystretch}{1.4}
		\begin{longtable}[c]{ l l >{\raggedright}p{5cm} >{\raggedright}p{3cm} l }
			\textbf{Version} & \textbf{Datum} & \textbf{Utförda förändringar} 
			& \textbf{Utförda av} & \textbf{Granskad} \\ \midrule
			
			1.0 & 2016--12--20 & Första versionen & Projektgruppen &
			Projektgruppen \\
			
		\end{longtable}
	\end{center}
	
	%%%%%%%%%%%%%%%%%%%%%%%%%%%%%%%%%%%%%%%%%%%%%%%%%%%%%%%%%%%%%%%%%%%%%%%%%%%%%%%%%
	%						Inledning
	%%%%%%%%%%%%%%%%%%%%%%%%%%%%%%%%%%%%%%%%%%%%%%%%%%%%%%%%%%%%%%%%%%%%%%%%%%%%%%%%%
	
	\newpage
	
	\raggedright
	
	\section{Tidsåtgång}
    Tidsåtgången för projektet har varit stor. Alldeles för mycket av den tiden
    spenderades i slutet av projektet.
	
	\subsection{Arbetsfördelning}
    Alla i gruppen har lagt ner minst 170 timmar, men vissa har fått arbeta
    lite mer, dels på grund av specifika hårdvarufel samt dålig fördelning av
    uppgifter.
	
	\subsection{Tidsåtgång jämfört med planerad tid}
    Den planerade tidsåtgången för projektet var 1120 timmar, men 1329 timmar
    användes.
	
	\section{Analys av arbete och problem}
	
	\subsection{Vad hände under de olika faserna?}

    Före-fasen, alltså skriving av kravspecifikation och projektplan, gick utan
    problem. Vi skulle dock i skrivandet av projektplanen lagt mer arbete på
    planering av fördelning av arbetsuppgifter. Exempelvis skulle vi ha
    analyserat den kritiska vägen bättre, samt undvika situationer där det är
    bara en eller två personer som känner till de tekniska detaljerna för ett
    visst område.

    Skrivningen av designspecifikationen under Under-fasen gick bra. Vi
    skulle dock ha fokuserat mer på kommunikation mellan personer som arbetade på
    olika system, för att undvika duplicerat arbete och oenighet om
    koordinatsystem.

    Under implementationsfasen av projektet, hade vi mycket hårdvarufel och
    problem vid integration av enheterna. Vi fick också problem med att roboten
    inte kunde gå förrän sent i projektet, vilket gjorde det svårare för andra
    enheter att testas ordentligt, t ex beslutsfattning och korridorsföljning.
    Detta berodde på vår dåliga planering där motorikenheten enbart tilldelades
    till två personer, vilket möjligtvis hade kunnat undvikas med bättre
    kritisk-väg-analys.

    Under Efter-fasen skulle vi ha lagt ner mer tid på presentationen, så att
    folk kunde öva sina delar mer.
	
	\subsection{Hur vi arbetade tillsammans}
    Vissa parprogrammerade, medan vissa arbetade själva med en komponent av
    roboten. De som inte parprogrammerade skulle ha haft en bättre
    kommunikation med resten av gruppen.
	
	\subsection{Hur använde vi projektmodellen?}
    Vi följde projektmodellen ganska bra. Vi utvecklade systemen, enhetstestade
    och sedan integrationstestade dem, enligt V-modellen.
	
	\subsection{Hur fungerade relationen med beställaren?}
    Relationen med beställaren har gått bra. Tidrapporter har skickats i tid
    varje vecka och statusrapporter har skickats på begäran. 
	
	\subsection{Hur fungerade relationen med handledaren?}
    Till en början använde vi inte handledaren särskilt mycket, vilket vi
    skulle ha gjort. I slutet, då alla hårdvaruproblem uppstod, tog vi ganska
    mycket råd och hjälp från handledaren.
	
	\subsection{Tekniska problem}

    Nedan beskriver vi de tre mest tidskrävande och svårlösta buggar, rankade
    med 1 som mest tidskrävande, och 3 minst.

    \subsubsection{Problem 1: Instabil SPI-buss}

    Detta hårdvarufel uppkom i slutet av vecka 49. SPI-kommunikationen visade
    sig vara oerhört instabil när både motorikenheten och sensorenheten
    användes tillsammans på bussen. Undersökning med oscilloskåp visade att
    signalerna hade mycket störningar, som gjorde kommunikationen svår,
    särskilt till sensorenheten.

    Lösningen på detta problem var helt enkelt att korta ner kablarna som
    används i SPI-bussen, så att de blev så korta som möjligt.

    Hade korta och mer strukturerat virade kablar använts, hade runt 56 timmar
    sparats. % TODO check

    \subsubsection{Problem 2: Konstiga svar från servon}

    Detta var ett hårdvarufel som hade två huvudsymtom. Signalen på servobussen fick 
	konstiga värden där den digitala signalen varierade mellan tre nivåer. Senare i 
	projektet upptäckte vi även att roboten inte kunde svänga runt sin egen axel och
	inte heller kunde gå rakt åt sidan. 
	
	Till slut insåg vi att problemet berodde på att två servon hade fått samma ID. 
	När man begärde data från de servona försökte båda svara samtidigt och signalen
	blev fel.

    Ungefär 30 timmar gick åt att försöka åtgärda detta fel.

    \subsubsection{Problem 3: Motorikenheten startade ständigt om}

    Detta mjukvarufel uppstod ganska ofta på motorikenheten.
    Roboten kunde inte gå en längre tid utan att programmet på motorikenheten
    startade om från början, vilket gjorde att roboten lade sig ner på golvet
    och ställde sig upp igen. Orsaken visade sig vara ett minnesfel i
    SPI-kommunikationen, då en buffer felindexerades när längd-byten av
    meddelandet blev fel. Lösningen var dels att åtgärda en bugg som
    gjorde att enheten behandlade meddelanden som paritetskontrollen visat vara
    fel, samt att kontrollera att längd-byten inte visar en omöjlig längd på
    meddelandet.

    Det är svårt att uppskatta hur mycket tid som gick förlorat på grund av
    denna bugg, men minst 20 timmar har lagts ner på att leta efter minnesfel i
    motorikkoden.


	\section{Måluppfyllelse}
	
	\subsection{Vad har uppnåtts?}
    Alla 1-prioritetskrav i kravspecifikationen har uppnåtts. Vi har dessutom
    mer avancerad gångstil än vad som krävdes, och roboten kom tvåa i både den
    manuella och autonoma tävlingen.
	
	\subsection{Hur fungerade leveransen?}
    Leveransen blev fördröjd i någon timme, på grund av ett servofel. När
    roboten testades efter åtgärdning av felet uppfyllde den dock alla krav.

    Leveransen av vår dokumentation gick halvbra, eftersom beställaren först
    inte kunde öppna den tekniska dokumentationen på sin dator. Problemet
    löstes dock dagen efter.
	
	\subsection{Hur har studiesituationen påverkat projektet?}
    Det är snarare projektet som påverkat studiesituationen, då vi har hamnat
    lite efter i de andra kurserna för att få roboten att fungera.
	
	\section{Sammanfattning}
    Vi anser att projektet har gått bra, då vi lyckades leverera en fungerande
    robot, samt har lärt oss väldigt mycket om robotteknik och projektarbeten i
    allmänhet.
	
	\subsection{De tre viktigaste erfarenheterna}
    \begin{itemize}
        \item Det är viktigt att integrera delsystemen tidigt.
        \item Det är bra att se till så fler personer förstår flera delsystem.
        \item Det är bättre att göra hårdvaran (virning etc) ordentligt från början, istället
            för att försöka förbättra den i efterhand.
    \end{itemize}
	
	\subsection{Goda råd till de som ska utföra ett liknande projekt}
    \begin{itemize}
        \item Använd PWM på LIDAR-sensorn, om ni endast ska mäta avstånd med
            den.
        \item Ha så korta och snyggt virade kablar som möjligt i digitala
            bussar.
        \item Skriv modulär kod som är enkel att modifiera.
        \item Lägg mycket tid i början av projektet så att ni slipper lägga ner
            väldigt mycket tid i slutet.
        \item Gör simulatorer.
    \end{itemize}

	
\end{document}
